\documentclass[12pt]{ctexart}
\usepackage{geometry}
\geometry{a4paper,margin=2.5cm}
\usepackage{setspace}
\onehalfspacing
\begin{document}

\title{人的有限性及其超越}
\date{\today}
\maketitle

\section*{引言}
开场\par
我们常谈目标、计划与未来,但很少直接谈一个最根本的事实:每个人都有限。有限不是遥远的哲学概念,而是每天在我们决策、焦虑、欲望中显现的底色。\par
今天我们把话题分成五部分:认识有限、看清逃避、觉醒向死、为己与为他、以及最后的行动呼吁。每一部分都尽量清晰、可记,并留给大家具体的思考与实践建议。\par

\section*{有限性一:死亡是人的基本规定}
死亡并非可有可无的选项,而是人的终极边界。它为生命设定了时限,也为我们选择的每一步带来紧迫感。\par
厄内斯特·贝克尔指出,死亡意识深刻驱动着宗教、艺术、科学与政治等人类活动。我们争名、争利、追求成就,往往是希望在有限里留下某种延续。\par
这种延续欲本身既有正面意义,也藏着恐惧:我们害怕被时间抹去,害怕我们的生活没有被看见、没有影响力。\par
理解死亡的必然性,能帮助我们分辨:哪些努力是真正为了热爱与责任,哪些只是为掩饰内心的不安与空虚。\par

\section*{有限性二:常见的逃避与代偿}
面对死亡,人类常走两条逃避之路:寻求彼岸救赎,或在此岸制造忘却。\par
宗教提供安慰與永生的想象;现代社会则以科学、消费与忙碌安放我们的焦虑。二者都有可能变成麻痹工具,而非面对现实的勇气。\par
还有更微妙的逃避:父母将未竟的愿望强加给孩子,社会用“成功学”掩盖价值的虚无,极限行为以“和死擦肩而过”的刺激来证明存在。\par
这些代偿可能带来短暂的满足,但长期会剥夺珍惜当下的能力,使我们越来越难以承受真正的自由与责任。\par

\section*{超越一:觉醒——向死而生的存在态度}
“超越”并非消灭有限,而是在有限中活出更真实的生命。海德格尔提出的“向死而在”就是把死亡作为存在的可能性来把握。\par
把死亡放进日常思考,不是沉湎恐惧,而是把终点作为行动与选择的坐标。当我们以终点为参照,很多琐碎的焦虑会逐渐清晰:什么值得做,什么可以放手。\par
东方与西方的智慧都有方法帮助我们觉醒:庄子提醒我们与命运和解,伊壁鸠鲁教我们认识感受的界限,从而减少无谓的恐惧。\par
实践上,这可以表现为写“遗愿清单”、设立长期目标、定期反省人生优先级—这些都是把死亡变成行动资源的练习。\par

\section*{超越二:为己与为他——双重的实践路径}
超越有限性有两条互补路径:为己与为他。\par
为己,意味着自我修炼:戒除拖累自己的坏习惯、培养觉察力、学习把当下活得更有深度。这是以内在的完整回应有限。\par
为他,则是把个人生命连接到他人的福祉:通过关怀、教育、公共服务或创造性工作,让你的努力在他人生命中留下痕迹,从而延伸你的意义。\par
两者并非对立,而是互为强化:为己使你在行动时更诚实,为他让你的生命有更持久的回响。\par
在现实选择上,把“要不要读研、如何择业、如何对待亲情”与对死亡的清醒对话结合,你的步骤就不会只是被焦虑驱动,而是被意义引导。\par

\section*{结语与行动呼吁}
我们注定有始有终,但可以选择带着更少的悔恨走完这段路。\par
请问自己三个简短的问题:我真正在怕什么?我为谁而活?我的今天会让未来的我感到骄傲吗?\par
从小事开始实践:戒掉一项拖累你的习惯;每天对一位重要的人表达感谢;给十年后的自己写一封信并保留;每个月做一件真正想做的事而不是被迫的事。\par
这些简单的行动,就是向死而生的练习。愿我们在有限中发现更广阔的意义,在有限里完成我们的超越。谢谢大家。\par

\end{document}